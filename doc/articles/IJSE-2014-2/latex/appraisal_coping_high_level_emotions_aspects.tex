\documentclass{apa6e}
\usepackage{apacite} % This is not done automatically!
\usepackage{graphicx}
\bibliographystyle{apacite}
\title{Appraisal, coping and high level emotions aspects of computational emotional thinking.}
\shorttitle{Computational appraisal, coping and high level emotions.}
\author{Max ~Talanov and Alexander ~Toschev\\Higher Institute of Information Technologies and Information Systems of \\Kazan Federal University}
\authornote{Correspondence concerning this article should be addressed to Max Talanov,
e-mail: max.talanov@gmail.com or Alexander Toschev, e-mail: alexander.toshchev@gmail.com}
\abstract{Turing genius anticipated current research in AI field for 65 years and stated that idea of intelligent machines "cannot be wholly ignored, because the idea of 'intelligence' is itself emotional rather than mathematical" \cite{intelligent_machinery}.
This is second article dedicated to emotional thinking bases. In first article we \cite{computational_emotional_thinking} created overall picture and proposed framework for computational emotional thinking. We used 3 bases for our work: AI - six thinking levels model described in book "The emotion machine" \cite{emotionmachine}. Evolutionary psychology model: "Wheel of emotions" \cite{natureofemotions}. Neuroscience (neurotransmission) theory of emotions by L\"{o}vheim "Cube of emotions" \cite{cubeofemotions}. Based on neurotransmitters impact we proposed to model emotional computing systems. Current work is dedicated to three aspects left not described in first article: \emph{appraisal}: algorithm and predicates - how inbound stimulus is estimated to trigger proper emotional response, \emph{coping}: the way human treat with emotional state triggered by stimulus appraisal and further thinking processes, \emph{high level emotions} impact on system and its computational processes.\\
Keywords: AI, Cognitive architecture, Cognitive and Affective Modeling, Machine Thinking, Machine Understanding, Computing Emotions, Affective Computing, Neuromodulation, Neurotransmission, Model of Emotions, Model of Emotional Feedback Loops}
\begin{document} 
\maketitle
\section{Introduction}

There are three bases of computational emotions thinking theory presented in our previous\cite{computational_emotional_thinking} and current article: neuroscience: \cite{emotionsbraintorobot, parsingreward, neuromodulatory, cubeofemotions, natureofemotions}, to computer science: \cite{emotionandsociable, senticcomputing, hourglass, affectivemodelofinterplay, affectivecomputing, computationalmodelsemotion, computationalmodelsemotionscognition, evaluatingcomutationalmodel, threelevel} and evolutional psychology: \cite{natureofemotions, primer_affect_psychology, tomkins1, tomkins2, tomkins3, quest}.

Overall emotional process was described exhaustively in our previous article \cite{computational_emotional_thinking} and looks like following:

\begin{enumerate}
	\item  Inbound stimulus is appraised non-consciously (affective appraisal)
	\item  Neuromodulation is triggered, it actually switches the emotional state of the system. System feels emotion
	\item  Conscious processes are triggered: stimulus cognition with stimulus deliberation, stimulus reflective thinking, stimulus cognition reflection, stimulus cognition self-reflection, stimulus cognition self-conscious reflection (cognitive appraisal)
	\item  Parallel to conscious processes the instinctive behaviour could be triggered, it influences the environment
	\item  Conscious processes described above triggers conscious behaviour and its turn it influences the environment again
\end{enumerate}

Neuromodulators influence in emotional processes was described in Lovheim article \cite{cubeofemotions}. We used Plutchik "wheel of emotions" as base psychological model of emotions \cite{natureofemotions} and adopted his emotional feedback loops processes to fit cognitive architecture "model of six" of Marvin Minksy \cite{emotionmachine}. We developed mapping of \emph{neuromodulators impact over computational processes}:

\begin{enumerate}
	\item  Generic:
	\begin{enumerate}
		\item  CPU power: noradrenaline
		\item  Memory distribution (attention): noradrenaline
		\item  Learning: serotonin, dopamine
		\item  Storage: serotonin, dopamine
	\end{enumerate}
	\item  Decision making/reward processing:
	\begin{enumerate}
		\item  Confidence: serotonin
		\item  Satisfaction: serotonin
		\item  Motivation, wanting: dopamine
		\item  Risky choices inclination: noradrenaline
		\item  Number of options to process: noradrenaline
	\end{enumerate}
\end{enumerate}

Thus we identified psychological phenomena of emotions influence on computational processes. This could be used as base for further computational emotional thinking framework. We defined overall architecture of emotional processes but we left several aspects not described for further research. Current article is dedicated to cover those blank spots: emotional appraisal (non-conscious and conscious), coping and influence of high level emotions on computational processes.


\bibliography{ref}
\section{Appendix}

\subsection{Comprehensive SECs structure}

\begin{enumerate}
 \item  Relevance detection:
 \begin{enumerate}
  \item  Novelty check:
  \begin{enumerate}
   \item  Suddenness
   \item  Familiarity
   \item  Predictability
  \end{enumerate}
  \item  Intrinsic pleasantness check
  \item  Goal relevance
 \end{enumerate}
 \item  Implication assessment:
 \begin{enumerate}
  \item  Causal Attribution check: intentional(int)/other/natural(nat)/negligence(neg)/chance(ch)
  \item  Outcome probability check
  \item  Discrepancy from expectation check: consonant/dissonant
  \item  Goal/need conduciveness check: high/obstruct
  \item  Urgency check
 \end{enumerate}
 \item  Coping potential determination
 \begin{enumerate}
  \item  Control check
  \item  Power check
  \item  Adjustment check
 \end{enumerate}
 \item  Normative significance evaluation:
 \begin{enumerate}
  \item  Internal standards check
  \item  External standards check
 \end{enumerate}
\end{enumerate}

\subsection{Conscious appraisal predicates}

\subsubsection{Joy}

\begin{enumerate}
 \item  Relevance detection:
 \begin{enumerate}
  \item  Novelty check:
  \begin{enumerate}
   \item  Suddenness. = m-h
   \item  Familiarity. = o
   \item  Predictability. = l
  \end{enumerate}
  \item  Intrinsic pleasantness check. = o
  \item  Goal relevance. = h
 \end{enumerate}
 \item  Implication assessment:
 \begin{enumerate}
  \item  Causal Attribution check
  \begin{enumerate}
   \item  Cause: agent. = o
   \item  Cause: motive. = ch
  \end{enumerate}
  \item  Outcome probability check. = vh
  \item  Discrepancy from expectation check. = o
  \item  Goal/need conduciveness check. = vh
  \item  Urgency check. = l
 \end{enumerate}
 \item  Coping potential:
 \begin{enumerate}
  \item  Control check. = o
  \item  Power check. = o
  \item  Adjustment check. = m
 \end{enumerate}
 \item  Normative significance evaluation:
 \begin{enumerate}
  \item  Internal standards check. = o
  \item  External standards check. = o
 \end{enumerate}
\end{enumerate}

\subsubsection{Anticipation}

\begin{enumerate}
 \item  Relevance detection:
 \begin{enumerate}
  \item  Novelty check:
  \begin{enumerate}
   \item  Suddenness. = m
   \item  Familiarity. = o
   \item  Predictability. = l
  \end{enumerate}
  \item  Intrinsic pleasantness check. = o
  \item  Goal relevance. = h
 \end{enumerate}
 \item  Implication assessment:
 \begin{enumerate}
  \item  Causal Attribution check
  \begin{enumerate}
   \item  Cause: agent. = o
   \item  Cause: motive. = int
  \end{enumerate}
  \item  Outcome probability check. = vh
  \item  Discrepancy from expectation check. = d
  \item  Goal/need conduciveness check. = h
  \item  Urgency check. = m
 \end{enumerate}
 \item  Coping potential:
 \begin{enumerate}
  \item  Control check. = o
  \item  Power check. = o
  \item  Adjustment check. = o
 \end{enumerate}
 \item  Normative significance evaluation:
 \begin{enumerate}
  \item  Internal standards check. = o
  \item  External standards check. = l
 \end{enumerate}
\end{enumerate}

\subsubsection{Anger}

\begin{enumerate}
 \item  Relevance detection:
 \begin{enumerate}
  \item  Novelty check:
  \begin{enumerate}
   \item  Suddenness. = l-h
   \item  Familiarity. = l
   \item  Predictability. = l-m
  \end{enumerate}
  \item  Intrinsic pleasantness check. = o
  \item  Goal relevance. = m-h
 \end{enumerate}
 \item  Implication assessment:
 \begin{enumerate}
  \item  Causal Attribution check
  \begin{enumerate}
   \item  Cause: agent. = o, other
   \item  Cause: motive. = int/neg
  \end{enumerate}
  \item  Outcome probability check. = vh
  \item  Discrepancy from expectation check. = d
  \item  Goal/need conduciveness check. = ob
  \item  Urgency check. = m-h
 \end{enumerate}
 \item  Coping potential:
 \begin{enumerate}
  \item  Control check. = h
  \item  Power check. = m-h
  \item  Adjustment check. = h
 \end{enumerate}
 \item  Normative significance evaluation:
 \begin{enumerate}
  \item  Internal standards check. = o
  \item  External standards check. = l
 \end{enumerate}
\end{enumerate}

\subsubsection{Disgust}

\begin{enumerate}
 \item  Relevance detection:
 \begin{enumerate}
  \item  Novelty check:
  \begin{enumerate}
   \item  Suddenness. = o
   \item  Familiarity. = l
   \item  Predictability. = l
  \end{enumerate}
  \item  Intrinsic pleasantness check. = vl
  \item  Goal relevance. = l
 \end{enumerate}
 \item  Implication assessment:
 \begin{enumerate}
  \item  Causal Attribution check
  \begin{enumerate}
   \item  Cause: agent. = o
   \item  Cause: motive. = o
  \end{enumerate}
  \item  Outcome probability check. = vh
  \item  Discrepancy from expectation check. = o
  \item  Goal/need conduciveness check. = o
  \item  Urgency check. = m
 \end{enumerate}
 \item  Coping potential:
 \begin{enumerate}
  \item  Control check. = o
  \item  Power check. = o
  \item  Adjustment check. = o
 \end{enumerate}
 \item  Normative significance evaluation:
 \begin{enumerate}
  \item  Internal standards check. = o
  \item  External standards check. = o
 \end{enumerate}
\end{enumerate}


\subsubsection{Sadness}

\begin{enumerate}
 \item  Relevance detection:
 \begin{enumerate}
  \item  Novelty check:
  \begin{enumerate}
   \item  Suddenness. = l
   \item  Familiarity. = l
   \item  Predictability. = o
  \end{enumerate}
  \item  Intrinsic pleasantness check. = o
  \item  Goal relevance. = h
 \end{enumerate}
 \item  Implication assessment:
 \begin{enumerate}
  \item  Causal Attribution check
  \begin{enumerate}
   \item  Cause: agent. = o
   \item  Cause: motive. = cha/neg
  \end{enumerate}
  \item  Outcome probability check. = vh
  \item  Discrepancy from expectation check. = o
  \item  Goal/need conduciveness check. = ob
  \item  Urgency check. = l
 \end{enumerate}
 \item  Coping potential:
 \begin{enumerate}
  \item  Control check. = vl
  \item  Power check. = vl
  \item  Adjustment check. = m
 \end{enumerate}
 \item  Normative significance evaluation:
 \begin{enumerate}
  \item  Internal standards check. = o
  \item  External standards check. = o
 \end{enumerate}
\end{enumerate}

\subsubsection{Surprise}

\begin{enumerate}
 \item  Relevance detection:
 \begin{enumerate}
  \item  Novelty check:
  \begin{enumerate}
   \item  Suddenness. = m
   \item  Familiarity. = l
   \item  Predictability. = vl
  \end{enumerate}
  \item  Intrinsic pleasantness check. = vl
  \item  Goal relevance. = h
 \end{enumerate}
 \item  Implication assessment:
 \begin{enumerate}
  \item  Causal Attribution check
  \begin{enumerate}
   \item  Cause: agent. = oth/nat
   \item  Cause: motive. = cha/neg
  \end{enumerate}
  \item  Outcome probability check. = vh
  \item  Discrepancy from expectation check. = d
  \item  Goal/need conduciveness check. = ob
  \item  Urgency check. = m
 \end{enumerate}
 \item  Coping potential:
 \begin{enumerate}
  \item  Control check. = vl
  \item  Power check. = vl
  \item  Adjustment check. = m
 \end{enumerate}
 \item  Normative significance evaluation:
 \begin{enumerate}
  \item  Internal standards check. = o
  \item  External standards check. = o
 \end{enumerate}
\end{enumerate}

\subsubsection{Fear}

\begin{enumerate}
 \item  Relevance detection:
 \begin{enumerate}
  \item  Novelty check:
  \begin{enumerate}
   \item  Suddenness. = h
   \item  Familiarity. = l
   \item  Predictability. = l
  \end{enumerate}
  \item  Intrinsic pleasantness check. = l
  \item  Goal relevance. = h
 \end{enumerate}
 \item  Implication assessment:
 \begin{enumerate}
  \item  Causal Attribution check
  \begin{enumerate}
   \item  Cause: agent. = oth/nat
   \item  Cause: motive. = o
  \end{enumerate}
  \item  Outcome probability check. = h
  \item  Discrepancy from expectation check. = d
  \item  Goal/need conduciveness check. = ob
  \item  Urgency check. = vh
 \end{enumerate}
 \item  Coping potential:
 \begin{enumerate}
  \item  Control check. = o
  \item  Power check. = vl
  \item  Adjustment check. = l
 \end{enumerate}
 \item  Normative significance evaluation:
 \begin{enumerate}
  \item  Internal standards check. = o
  \item  External standards check. = o
 \end{enumerate}
\end{enumerate}

\subsubsection{Trust}

\begin{enumerate}
 \item  Relevance detection:
 \begin{enumerate}
  \item  Novelty check:
  \begin{enumerate}
   \item  Suddenness. = h
   \item  Familiarity. = vl
   \item  Predictability. = l
  \end{enumerate}
  \item  Intrinsic pleasantness check. = vl
  \item  Goal relevance. = h
 \end{enumerate}
 \item  Implication assessment:
 \begin{enumerate}
  \item  Causal Attribution check
  \begin{enumerate}
   \item  Cause: agent. = oth/nat
   \item  Cause: motive. = ch
  \end{enumerate}
  \item  Outcome probability check. = vh
  \item  Discrepancy from expectation check. = c
  \item  Goal/need conduciveness check. = vh
  \item  Urgency check. = m
 \end{enumerate}
 \item  Coping potential:
 \begin{enumerate}
  \item  Control check. = o
  \item  Power check. = vl
  \item  Adjustment check. = m
 \end{enumerate}
 \item  Normative significance evaluation:
 \begin{enumerate}
  \item  Internal standards check. = o
  \item  External standards check. = o
 \end{enumerate}
\end{enumerate}

\end{document}
