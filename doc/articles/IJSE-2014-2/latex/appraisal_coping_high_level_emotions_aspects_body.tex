\section{Introduction}

There are three bases of computational emotions thinking theory presented in our previous\cite{computational_emotional_thinking} and current article: neuroscience: \cite{emotionsbraintorobot, parsingreward, neuromodulatory, cubeofemotions, natureofemotions}, to computer science: \cite{emotionandsociable, senticcomputing, hourglass, affectivemodelofinterplay, affectivecomputing, computationalmodelsemotion, computationalmodelsemotionscognition, evaluatingcomutationalmodel, threelevel} and evolutional psychology: \cite{natureofemotions, primer_affect_psychology, tomkins1, tomkins2, tomkins3, quest}.

Overall emotional process was described exhaustively in our previous article \cite{computational_emotional_thinking} and looks like following:

\begin{enumerate}
	\item  Inbound stimulus is appraised non-consciously (affective appraisal)
	\item  Neuromodulation is triggered, it actually switches the emotional state of the system. System feels emotion
	\item  Conscious processes are triggered: stimulus cognition with stimulus deliberation, stimulus reflective thinking, stimulus cognition reflection, stimulus cognition self-reflection, stimulus cognition self-conscious reflection (cognitive appraisal)
	\item  Parallel to conscious processes the instinctive behaviour could be triggered, it influences the environment
	\item  Conscious processes described above triggers conscious behaviour and its turn it influences the environment again
\end{enumerate}

Neuromodulators influence in emotional processes was described in Lovheim article \cite{cubeofemotions}. We used Plutchik "wheel of emotions" as base psychological model of emotions \cite{natureofemotions} and adopted his emotional feedback loops processes to fit cognitive architecture "model of six" of Marvin Minksy \cite{emotionmachine}. We developed mapping of \emph{neuromodulators impact over computational processes}:

\begin{enumerate}
	\item  Generic:
	\begin{enumerate}
		\item  CPU power: noradrenaline
		\item  Memory distribution (attention): noradrenaline
		\item  Learning: serotonin, dopamine
		\item  Storage: serotonin, dopamine
	\end{enumerate}
	\item  Decision making/reward processing:
	\begin{enumerate}
		\item  Confidence: serotonin
		\item  Satisfaction: serotonin
		\item  Motivation, wanting: dopamine
		\item  Risky choices inclination: noradrenaline
		\item  Number of options to process: noradrenaline
	\end{enumerate}
\end{enumerate}

Thus we identified psychological phenomena of emotions influence on computational processes. This could be used as base for further computational emotional thinking framework. We defined overall architecture of emotional processes but we left several aspects not described for further research. Current article is dedicated to cover those blank spots: emotional appraisal (non-conscious and conscious), coping and influence of high level emotions on computational processes.

