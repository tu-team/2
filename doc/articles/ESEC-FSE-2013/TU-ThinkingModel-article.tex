\documentclass[runningheads,a4paper]{llncs}

\usepackage{amssymb}
\setcounter{tocdepth}{3}
\usepackage{graphicx}
\usepackage{url}
\usepackage{epsfig}


\urldef{\mailalex}\path|{alexander.toschev@gmail.com}|
\urldef{\mailmax}\path|{max.talanov@gmail.com}|
\urldef{\mailand}\path|{andrey.krekhov@ts.fujitsu.com}|

\begin{document}

\mainmatter

\title{#Thinking model and machine understanding of English primitive texts and it's application in Infrastructure as Service domain.}

\titlerunning{Thinking Lifecycle implementation in ITSM}

\author{Alexander Toschev\inst{1} \and Maxim Talanov\inst{2} \and Andrey Krekhov\inst{3}}
\institute{Kazan State University, Chebotarev Research Institute of Mathematics and Mechanics\\
Universitetskaya 17, 420008 Kazan, Russia\\
\mailalex\\
\and
Fujitsu GDC Russia, Kazan, Russia\\
\mailmax\\
\and
Fujitsu GDC Russia,\\
Sibirskii trakt 34, 420029 Kazan, Russia\\
\mailand\\
}


\maketitle

\begin{abstract}

Construction of machine understanding is definitely the challenge. There are several technologies used widely.
Currently mainstream applications uses machine operatable knowledge bases, for example Wolfram Alpha to support simple dialog and operate devices.
Newer the less those approaches do not answer the question how do machine could be capable to understand human created text.
We tried new approach based on assumption that human understanding is tightly coupled with human thinking itself.
We used thinking model described in Marvin Minsky book "The emotion machine"\cite{minsk}.

\end{abstract}

\section{The emotion machine thinking model}
\subsection{6 thinking levels}
\subsection{Selector, Critic, Way to think triple}

\section{Implementation of thinking model}
\subsection{Thinking levels control}
\subsection{Memory: short term, long term}


\section{IS domain application of the thinking model}
\subsection{Critics and ways to think for incident processing}

\section{Practical results}
\subsection{Direct instruction processing}
\subsection{Problem description processing}


\begin{thebibliography}{4}

\bibitem{minsk}
Minsky M.:
The Emotion Machine.
Simon \& Schuster Paperbacks  (2006).

\bibitem{lili}
Liu H., Lieberman H.:
Metafor: Visualizing Stories as Code.
Cambridge, MIT Media Laboratory  (2005).

\bibitem{runo}
Russel S., Norvig P.:
Artificial Intelligence. A Modern approach.
Pearson (2010).

\end{thebibliography}

\end{document}